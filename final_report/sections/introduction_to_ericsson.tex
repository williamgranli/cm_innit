\documentclass[fina_report_innit.tex]{subfiles}
\begin{document}

\section{Introduction to Ericsson}

\subsection*{General}
[Ericsson]
Ericsson operates in the domain of communications technology, and their “main” products are in the domains of mobility, broadband and cloud-based services [1] [2]. Ericsson has 117,508 employees (as of September 30, 2014), 25,300 of which work in R\&D (Research and Development) [1].

[EPG]
Our research has been conducted at the EPG (Evolved Packet Gateway) department of Ericsson. The EPG department consists of circa 700 employees, and the larger majority of them work with software development-related tasks [4]. [[informal?]]  The EPG department delivers and maintains the EPG product which is a part of Ericsson’s product for deployment of LTE (Long Term Evolution) networks [3]. The EPG product serves as a gateway between Ericsson’s packet core network and other packet data networks, such as the Internet [3]. The EPG product is utilized in technologies such as 2G, 2G+, 3G, 4G and Wi-Fi hotspots [3].

\subsection*{MDD}

[MDD vs Manual at Ericsson/EPG]
The numerous subsystem that make up the EPG product are both developed through [[through, with, in?]] manual coding and MDD. The distribution between manual coding and MDD in EPG differs depending on how it is measured. According to our interview [[with Jesper?]] the distribution of employees contributing (through commits) to the code base, the distribution is close to 1:1 [4]. However, the number of teams developing [[through]] manual coding and MDD is circa 1:7/3 [4]. The distribution of requirements that relate to manual code and MDD code is also about 1:7/3 [4].

[Introduction of MDD at Ericsson]
The introduction of MDD in Ericsson happened [[correct wording?]] in [[year]]. Due to a discontent with the quality of the current code base, a goal [[incorrect word]] of change spread within Ericsson [4]. The main factor which spurred the decision to re-architect the system relates to TRs (Trouble Reports, aka bugs [[]]) [4]. When TRs were resolved, more TRs would arise, which lead to an endless cycle of fixing TRs [4]. This characteristic of change is described in Van De Ven \& Poole’s teleological motor [8] which is change that is driven by the discontent with the current way of working and develops through the strive for improvement [8]. 

[new status quo]
The initial introduction of MDD at Ericsson mainly concerned changes on an architectural level (moving to a service-based architecture) and the introduction of new frameworks [4]. The general consensus, however, appears to be that a “fresh start” [[informal?]] was needed and that major changes were needed; even if it would be costly in terms of time and money [4]. According to the interview[[s]] the assumption also held at Ericsson was that, in order to do proper mordern software design MDD would need to be adopted [4]. The MDD tool that was chosen to be used, was IBM’s product; Rational Rhapsody [5].

[no subsystems changed]
To this date, no subsystems have been changed [[rewritten?]] from manual code to MDD code [4]. The parts of the EPG codebase that today are labeled as the MDD-part of the system have been created after the introduction of MDD, in [[year]] [4]. 

[hiring new employees]
The introduction [[repeating, look at 2 prev paragraphs]] of MDD did not only involve technological changes. Since MDD was a technology relatively unknown to Ericsson, many new employees were hired to help facilitate the change [4]. [[experts, kotter]] 

[split culture]
The hiring of new employees, and the fact that no parts of the Ericsson code base have been changed from manual code to MDD suggests that there may be a split in the employees’ regarding the change to MDD. [[technological frames??, culture??]]

[pros cons of MDD]
Most of the negative opinions about MDD in the EPG department of Ericsson relate to the tool [5] itself. [4] The most notable issue is the merging of different versions which is a time-consuming process with Rational Rhapsody [5]. This may be due to that most of the employees are accustomed to merging text-based files, and not graphical files such as those that compose the models in Rational Rhapsody [5] [4] [[how do i reference this? 5 points to rhapsody and 4 points to the whole sentence]]. The positive sides of the introduction of MDD relate to the modeling-part of MDD, and the knowledge sharing that visual models allows [4]. Another positive effect is the that MDD facilitates code reuse in a better way [4]. 



\end{document}
