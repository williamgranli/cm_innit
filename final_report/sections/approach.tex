\documentclass[final_report_innit.tex]{subfiles}
\begin{document}

\section{Research Approach}
Introduction is not included in draft.

\subsection*{Choice of research methodology}
	
To lay a foundation for our understandings of selected change areas at Ericsson, searches for related articles was done at engineeringvillage.com and ieeexplore.com. Compositions of search terms were; [[YAY DO WE REMEMBER WHAT WE SEARCHED FOR? LIKE `component testing + MDD']]. Further more, articles got a higher relevance score if they included ideas and conclusions about MDSD or Agile development. All articles do not discuss implementation in either an MDSD or manual coding environment, thus only providing understanding of the subject it self. Other articles are only relevant for either or. For example, component testing tend to be discussed related to a development process, whereas Cross-functional Teams is higly discussed as an beneficial factor for all kinds of development processes. [[Detta innebär således att...?]]

The research was split into two major parts; qualitative and quantitative.

\subsection*{Quantitative Research}

The quantitative part was based on one initial research question; Has there been differences in how changes have affected groups working with MDSD compared with groups programming manualy?	A quantitative data collection would expose correlations between the focus groups, thus implying if for example unit testing has been more appreciated in one group or another, but also to indicate that some process or design changes is not suited for either or. Therefore, questions suitable to answer on a scale from 1 to 5 was derived from our related articles. [[Say more about the questions? Or is that covered in result/analysis or Survey?]]

\subsection*{Qualitative Research}

As a qualitative addition, we conducted an interview to verify that data collected was valid. To better understand the quantitative data, we split the interview into two categories; one part verifying data, and one part focused on change implementation.

\end{document}