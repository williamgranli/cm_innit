\documentclass[final_report_innit.tex]{subfiles}
\begin{document}

\section{Research Approach}
Introduction is not included in draft.

\subsection{Choice of research methodology}
To lay the foundation for our understanding of the selected change areas at Ericsson, the search for related articles was done at engineeringvillage.com and ieeexplore.com. Compositions of search terms were; [[{},{},{},{},{},{}]]. Furthermore, articles received a higher relevance score if they included ideas and conclusions about MDSD or Agile development. Not every article discusses implementation in either an MDSD or manual coding environment, thus providing only understanding of the subject. Other articles are only relevant either for MDSD or manual coding. For example, component testing tends to be discussed in relation to a development process, whereas Cross-functional Teams are extensively discussed as a beneficial factor for all kinds of development processes.
\\*
\\*
The research was split into two major parts; qualitative and quantitative.
\\*
\subsubsection{Quantitative Research}
The quantitative part was based on one initial research question; Have there been differences in how changes have affected groups working with MDSD compared to groups programming manually? A quantitative data collection would expose correlations between the two groups, thus implying if for example, unit testing has been more applicable to one group or another, but also to indicate that some process or design changes is not suited for either or. Therefore, questions suitable to answer on a scale from 1 to 5 were derived from our related articles. The questions was shaped to embrace the following change areas;
\\*
\begin{enumerate}
	\item Continuous Integration,
	\item Cross-Functional Teams,
	\item One Track,
	\item Unit Testing, and
	\item Component Testing.
\end{enumerate}

\subsubsection{Qualitative Research}
As a qualitative addition, we conducted an interview to verify the collected data was valid. To better understand the quantitative data, we split the interview into two categories; one part verifying the data, and one part focused on change implementation.

\subsection{Survey}
With two research methods in mind, our survey likewise consisted of two data collections; a questionnaire and multiple interviews, for quantitative and qualitative data collection respectively. Whereas a scale from 1 to 5 was adequate to answer a few questions, some questions aimed to support a metric like Personal Fulfilment, would be interesting and insightful to discuss during an interview. Preferrably, interviews should be a larger part of the collected data to align with the reality at Ericsson. When forming our survey we considered the potential stress on Ericsson's employees from a large amount of interviews. Our contact person verified that employees could falsify answers (i.e. amongst many, answering without pondering the question) during interviews due to stress from leaving their obligations for too long. Additional interviews would possibly give us a more accurate reflection of the employees perception, however, a streamlined composition of close ended questions was formed to minimise the amount of time taken from participants.
\\*
\subsubsection{In-depth: Questionnaire}
To simplify the distribution and aligning with Ericsson's way-of-working, the questionnaire was created through Google Forms and distributed via e-mail. Google Sheets was used to analyse data, where we compared average, minimum, and maximum values for each question, and considering if the respondant worked with MDSD or manual coding. The sample group should need to have worked at Ericsson for at least five years, and had to have been a part of the EPG department. The constraints were necessary to collect responses from personnel who had experienced all change implementations mentioned in this research. Moreover, the result of the questionnaire aided us in pinpointing what to include in the interviews.
\\*
\subsubsection{In-depth: Interview}
The interview material was partly derived from a shorter analysis of the questionnaire results. That part's intention was to validate if the average, minimum, and maximum values were realistic. The intention was never to question the actual result, but rather making sure that our overall indications were believable to interviewees. The second part was intended to conclude if theories from the field of Change Management were applied, or if Ericsson had developed their own way of implementing change.
\\*
\\*
We interviewed a developer who also has extensive experience with software architecture. With deep understanding of both manual coding and achitectural descisions, as well as a good understanding of change implementation, he had the preferred pre-requisites for the interview.

\end{document}