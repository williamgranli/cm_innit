\documentclass[final_report_innit.tex]{subfiles}
\begin{document}

\section{Research Approach}
Introduction is not included in draft.

\subsection{Choice of research methodology}
To lay the foundation for our understanding of the selected change areas at Ericsson, the search for related articles was done at engineeringvillage.com and ieeexplore.com. Compositions of search terms were; [[YAY DO WE REMEMBER WHAT WE SEARCHED FOR? LIKE `component testing + MDD']]. Furthermore, articles received a higher relevance score if they included ideas and conclusions about MDSD or Agile development. Not every article discusses implementation in either an MDSD or manual coding environment, thus providing only understanding of the subject. Other articles are only relevant either for MDSD or manual coding. For example, component testing tends to be discussed in relation to a development process, whereas Cross-functional Teams are extensively discussed as a beneficial factor for all kinds of development processes. [[Detta innebär således att...?]]
\\*
\\*
The research was split into two major parts; qualitative and quantitative.

\subsubsection{Quantitative Research}
The quantitative part was based on one initial research question; Have there been differences in how changes have affected groups working with MDSD compared to groups programming manually? A quantitative data collection would expose correlations between the two groups, thus implying if for example, unit testing has been more appreciated in one group or another, but also to indicate that some process or design changes is not suited for either or. Therefore, questions suitable to answer on a scale from 1 to 5 were derived from our related articles. [[Say more about the questions? Or is that covered in result/analysis or Survey?]]

\subsubsection{Qualitative Research}
As a qualitative addition, we conducted an interview to verify the collected data was valid. To better understand the quantitative data, we split the interview into two categories; one part verifying the data, and one part focused on change implementation.

\subsection{Survey}
As for collecting data, we went with a two-sided approach. On one hand we will gather qualitative data through a questionnaire sent out to numerous employees at Ericsson. On the other hand our qualitative data collection is based on interviews with key personnel at Ericsson.

\subsubsection{Questionnaire}
The questionnaire is sent as a form to employees at Ericsson to answer. The preferred recipient works with EPG and has been with the company for at least 5 years. The questionnaire consists of questions related to CI, CFT, FT, UI, and CI. Four questions are posed for each subject and the recipient answers on a scale from 1-5 or inapplicable in case he or her has not worked with that particular field or practice.

\subsubsection{Interview}
Interview for mid level manager related to change. Interviewee central to changes and have an over head view of change - both process and motivations.
Questions related to motivations for change, and most of the questions are based off the results from the questionnaire. Get qualitative input on quantitative study.

\end{document}