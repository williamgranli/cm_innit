\documentclass[final_report_innit.tex]{subfiles}
\begin{document}

\section{Research and Survey Methodology}\label{approach}
In this section we shall cover how articles were selected, what parts the survey consists of, and in-depth discussions of how the survey was created and carried out. Section \ref{approachChoice} generally introduces article sources, on what basis some articles were discarded or used. Sections \ref{approachQuant} and \ref{approachQual} discusses why we performed quantitative and qualitative approaches respectively. Sections \ref{approachSurvey}, \ref{approachInQuest}, and \ref{approachInInt} goes further into explaining how the questionnaire and interview questions were motivated and formed.

\subsection{Motivation for Research Methodology}\label{approachChoice}
To lay the foundation for our understanding of the selected change areas at Ericsson, the search for related articles was done at engineeringvillage.com and ieeexplore.com. Compositions of search terms is listed in Appendix \ref{appendixSearchTerms}. Furthermore, articles received a higher relevance score if they included ideas and conclusions about MDSD or AD. Not every article discusses implementation in either an MDSD or manual coding environment, thus providing only a basic understanding of the subject. Other articles are only relevant either for MDSD or manual coding. For example, CT tends to be discussed in relation to a development process, whereas CFT are extensively discussed as a beneficial factor for all kinds of development processes.
\\

For these reasons the research was split into two major parts; quantitative and qualitative.
\\

\subsubsection{Quantitative Research}\label{approachQuant}
The quantitative part was based on one initial research question: Have there been differences in how changes have affected groups working with MDSD compared to groups programming manually? A quantitative data collection would expose correlations between the two groups, thus implying if, UT has been more applicable to one group or another, but also to indicate that some process or design changes are not suited for either. Therefore, questions suitable to answer on a scale from 1 to 5 were derived from our related articles. The questions were shaped to embrace the following change areas:
\\
\begin{enumerate}
	\item One Track,
	\item Cross-functional teams,
	\item Continuous integration,
	\item Unit testing, and
	\item Component testing. \\ % LEAVE LINE BREAK HERE (\\). NEEDED TO SEPARATE THE SECTIONS NUMBERING FOLLOWING DIRECTLY AFTER THE LIST.
\end{enumerate}

\subsubsection{Qualitative Research}\label{approachQual}
As a qualitative addition, we conducted an interview to verify the collected data. To better understand the quantitative data, we split the interview into two categories; one part verifying the data, and one part focusing on change implementation.

\subsection{Survey}\label{approachSurvey}
With two research methods in mind, our survey likewise consisted of two data collection techniques; a questionnaire and an interview, for quantitative and qualitative data collection respectively. Whereas a scale from 1 to 5 was adequate to answer a few questions, for it was not. For example, to support the metric of Personal Fulfilment, it would be interesting and insightful to discuss during an interview. Preferably, interviews should be a larger part of the collected data to align with the reality at Ericsson. When forming our survey we considered the potential stress on Ericsson's employees from a large amount of interviews. Our contact, Jesper Derehag, suggested that employees could falsify answers (i.e. amongst many, answering without pondering the question) during interviews due to stress from leaving their obligations for too long. 
\\

\subsubsection{Questionnaire}\label{approachInQuest}
To simplify the distribution and alignment with Ericsson's way-of-working, the questionnaire was created through Google Forms and distributed via e-mail. Google Sheets was used to analyse data, where the average, minimum, and maximum values for each question were compared, and considered if the respondant worked with MDSD or manual coding. The sample group must have worked at Ericsson for at least five years, and had to have been a part of the EPG. The constraints were necessary to collect responses from personnel who had experienced all change implementations mentioned in this research. Moreover, the result of the questionnaire aided us in forming the material for the interviews. 
\\

\subsubsection{Interview}\label{approachInInt}
The interview material was partly derived from a shorter analysis of the questionnaire results. That part's intention was to validate if the average, minimum, and maximum values were realistic. The intention was never to question the actual result, but rather making sure that our overall indications were believable to the interviewee. The second part was intended to conclude if theories from the field of Change Management were applied, or if Ericsson had developed their own way of implementing change.
\\

We interviewed a developer who has extensive experience with software architecture. With deep understanding of both manual coding and achitectural descisions, as well as a good understanding of change implementation, he had the preferred prerequisites for the interview.

\end{document}