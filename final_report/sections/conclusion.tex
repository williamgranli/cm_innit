\documentclass[fina_report_innit.tex]{subfiles}

\begin{document}
\section{Conclusion}
This report has aimed to determine the extent to which particular process and design activities apply to both the MDSD and manual coding paradigms. We have researched existing literature which relates to our question, in addition to conducting a case study at Ericsson EPG. Our research findings have been primarily based on survey data, which has aimed to answer to what extent MDSD and manual coding practitioners rate the performance, quality of deliverables, personal fulfillment and ease of implementation for each of the process and design activities. Furthermore, our interviewee at EPG has provided us with an insight into the changes occurring at the organisation. We found that these activities are overall applicable to both paradigms; however, the most notable exceptions relate to One Track and Unit Testing. In OT, manual coding practitioners revealed that the ease of implementation was a negative factor. We encourage further research to find the underlying cause for this particular difficulty. Conversely, MDSD practitioners rated the ease of implementation for UT unfavourably. Our interviewee made it evident that MDSD practitioners are moving away from UT, possibly based on the difficulty of implementation for UT. Our major limitations have been focusing on a single organisation,  the distribution of the survey questionnaires, and the lack of further interviews with organisational members. 


\subsection*{Acknowledgements}
We would like to thank Ericsson for their support, the staff who have been involved in our survey and finally our primary contact, who facilitated the survey process and provided us with the valuable insight into the changes at Ericsson EPG.

\end{document}