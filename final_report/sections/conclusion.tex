\documentclass[fina_report_innit.tex]{subfiles}

\begin{document}
\section{Conclusion}
This report has aimed to determine the extent to which process and design practices apply to both the MDSD and manual coding paradigms. We have researched existing literature which relates to our question, in addition to conducting a case study at Ericsson EPG. Our research findings have been primarily based on quantitative data, which has aimed to answer to what extent MDSD and manual coding practitioners rate the performance, quality of deliverables, personal fulfillment and ease of implementation for each of the process and design practices. We found that certain practices are not equally applicable to both paradigms, with respect to ease of implementation; One Track and unit testing. For One Track, the responses of manual coders to our questionnaire revealed that the ease of implementation was a negative factor. We encourage further research to find the underlying cause for this particular difficulty. Conversely, MDSD practitioners rated the ease of implementation for unit testing unfavourably. Our interviewee made it evident that MDSD practitioners are moving away from unit testing, possibly based on the difficulty of implementation. We therefore recommend that adopting practices for different paradigms should be carefully planned, with respect to implementation. Focusing on a single organisation has been one of our major limitations, along with the distribution of the questionnaires, and the lack of further interviews with organisational members. 


\subsection*{Acknowledgements}
We would like to thank Ericsson for their support, the staff who have been involved in our survey and finally our primary contact, Jesper Derehag, who facilitated the survey process and provided us with the valuable insight into the changes at Ericsson EPG.

\end{document}
