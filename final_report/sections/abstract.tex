\documentclass[final_report_innit.tex]{subfiles}

\begin{abstract}

Changing software processes of any kind, either by design or process activities, can be very challenging for any organisation to undertake. Over the years there have been a myriad of theories and papers which aim to help address the issue of change management. Ericsson is just one of a multitude of companies who have implemented changes to their software processes, due to the result of either bottom-up or external pressure. In this paper, we will use a case study of Ericsson’s implementation of change at the EPG department and more specifically in relation to how suitable the changes are to their existing development paradigms. This paper will combine a literature review of the changes that have been implemented, how the changes were implemented in relation to common change management theories and techniques, supplemented by both a quantitative and qualitative study. The research findings are then compared and contrasted against the results of the surveys. We found that, overall, the changes were positively received by respondents of the surveys, with certain exceptions. The limitation of working solely with EPG meant that the results we obtained would only be relevant to a small subset of Ericsson and could not be applied, without further research, to the rest of the company.

\end{abstract}
