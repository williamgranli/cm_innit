\documentclass[fina_report_innit.tex]{subfiles}

\begin{document}

\section{Related Work}

\blindtext

\subsection{Feature-driven Development and One Track}
In Xiaocheng et al.'s "Agile Development of Secure Web Applications" [adoswa] feature-driven development is introduced as a part of the process. Xiaocheng et al. specifically discuss agile development in the context of secure systems (defined as "a system that is protected against specific undesired outcomes") [adoswa] which is an attribute shared with Ericsson's system. Hribar et al. [ftriauotasld] have presented an overview of Ericsson's approach to One Track development which mainly is based on internal Ericsson documentation. Hribar et al. [ftriauotasld] introduce a set of main principles for One Track that are related to continuous integration and agile, and the focus is mainly set on how to successfully implement and develop according to the One Track development strategy. Although a large part of Ericsson's software is developed through MDSD, Hribar et al. do not identify any differences in how well One Track works in combination with MDSD compared to non-MDSD. Even though MDSD is common in the domain of telecommunications, the characteristics of a MDSD environment are not mentioned as important factors to consider in the litterature related to One Track and feature-driven development. 


\end{document}