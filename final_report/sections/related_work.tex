\documentclass[fina_report_innit.tex]{subfiles}

\begin{document}

\section{Related Work}


\subsection{One Track}
Hribar et al. \cite{hribar2008first} have presented an overview of Ericsson's approach to One Track development which is mainly based on internal Ericsson documentation. Hribar et al. introduce a set of main principles for One Track in relation to continuous integration and agile, with the focus mainly set on how to successfully implement and develop with the One Track development strategy \cite{hribar2008first}.

Despite the fact that a large part of Ericsson's software is developed with MDSD, Hribar et al. do not identify the extent to which One Track functions in combination with MDSD compared to MC. The characteristics of MDSD are not explored in One Track related literature, to the best of our knowledge.

\subsection{Cross-functional Teams}
Ghobadi describes Cross-functional Teams (CF) as people from different subunits of an organisation, with competing social identities and obligations. He continues to explain that such teams shall allow departmental or functional boundaries capable of overlooking bureaucracy and assure project completion \cite{ghobadi2011challenges}.

Cross-functionality may exist within organisations without the intention to value opinions equally and communicate as CF, as groups spanning between subunits often exist without explicitly stating so. However, for established CF, it is vital to equally consider constraints, benefits and obligations from all subunits \cite{ghobadi2011challenges}.

Moreover, Ghobadi elaborates the importance of team leaders, who communicate and moderate their teams, to effectively handle and benefit from emotional conflicts, misunderstandings on tasks, or interdepartmental disagreements \cite{ghobadi2011challenges}.

Marchwinski and Mandzuik similarly describe CF as ``Those individuals within the firm whose competencies are essential in achieving an optimal evaluation, with one core member from each primary function of the organisation'' \cite{marchwinski2000technical}. Furthermore, they state that teams are formed either proactively or reactively to address specific issues, and also that the key determinant of CF is a broad communication and interaction across departments \cite{marchwinski2000technical}.

\subsection{Unit Testing}
Unit testing is not an exclusive practice of any development paradigm, but is an important part of software development. While not limited to model-driven development, the trend of modeling software provides developers and testers with an excellent opportunity to lean on test generation technology to automatically generate conformance tests from UML diagrams \cite{mussa2009survey}, \cite{hartmann2004uml}.  

Cheon et al. claim traditional unit testing is a tedious and cumbersome process, as every method needs to be tested and changes to code may require changes to each test. The test code itself needs to be examined, and one could argue that test code is as likely to be as faulty as the code it is testing \cite{cheon2002simple}. As a solution, Cheon et al. propose unit test automation where possible, by using frameworks which encourage a closer integration of testing. Even with testing frameworks, however, unit testing is often a time consuming task that requires test cases to be manually coded and maintained when the code under test changes. This could discourage testing as a parallel task to coding \cite{cheon2002simple}. 

Model Driven Development offers another take on unit testing; Mussa et al. state that the use of UML has led many researchers to use state machine diagrams and sequence diagrams to generate tests, including unit tests \cite{mussa2009survey}. The advantage of model based testing is that it can easily be automated and shifted to earlier stages of the development process. Model based test generation allows generation of tests that are independent of other components of the system, aiding the process of unit testing \cite{javed2007automated}.

\subsection{Component Testing}
While not a concrete practice of MDSD, component testing can be an important part of software development, especially when it comes to large and complex systems which are comprised of many components.

Gao and Jerry describe problems with standard component testing \cite{gao2000component}. They state that making the tests reusable is the greatest challenge when using this type of testing. If the tests are created as part of a framework, they become dependent upon that framework. As an alternative, Gao and Jerry suggest using a technique referred to as Built-In-Tests (BIT) \cite{gao2000component}. BIT is an alternative to using a framework as involves the use of writing tests into the classes, modules and components themselves. It tends to be cheaper and faster to develop while not requiring an external suite or framework to run; the tests are instead accessed with an interface into the appropriate component.

During 2004-2006, Eldh et al. performed research in conjunction with Ericsson to help investigate faults or Trouble Reports (TRs) and their frequencies in a large, complex system \cite{eldh2007component}. They found that designers at component testing levels do not find faults related to unclear specifications \cite{eldh2007component}. These faults made up 46.6\% of all software faults discovered in that specific case study \cite{eldh2007component}. These types of faults are therefore discovered later in the testing process. The key reason they give is that it is very hard for a developer to design, implement and specify code given the complexity of the environment and the knowledge of the context which is required \cite{eldh2007component}. 

Eldh et al. found that during the year 2005, where there was a strong influence on component testing, quality drastically improved in the software, especially in terms of faults \cite{eldh2007component}. They ultimately came to the conclusion that unit and component testing have next to no chance of finding a majority of faults which stem from a lack of knowledge and context \cite{eldh2007component}. They further concluded that component testing is limited in very complex systems and it highlights the need for higher level integration and systems testing along side component testing \cite{eldh2007component}.

\subsection{Continuous Integration}
Continuous Integration (CI) was suggested by Grady Booch \cite{booch2006object}. In traditional software processes, integration would be performed sporadically throughout a project to mark the end of an iteration, or to finish a prototype. In a typical worst case scenario, integration would take place all at once, in a “big bang” method. Such approaches could be problematic, because there is a high risk that different components or modules would require extensive rework to integrate successfully.

Martin Fowler has defined CI as “a software development practice where members of a team  integrate  their  work  frequently,  usually  each person  integrates  at  least  daily  -  leading  to  multiple integrations per day. Each integration is verified by an automated build (including test) to detect integration errors as quickly as possible” \cite{fowler2006continuous}.

This definition tells us that daily integration, automated builds and automated testing are all part of CI. It is therefore crucial for an organization to have the appropriate frameworks prepared before starting CI. To the best of our knowledge, there is no research on CI in MDSD in particular, however CI is one of the core practices of agile.

\subsection{Impact of Agile Development on MDSD}

In our research, we measure the applicability of certain agile practices, namely continuous integration and One Track. Agile Development (AD) and MDSD both address the issues of high change rates, time-to-market, increased return on investment (ROI) and high quality software \cite{stojanovic2003component}. However, the solutions that are proposed by AD differ to those proposed by MDSD \cite{stojanovic2003component}. Since AD has been developed to suit the object-oriented paradigm, Stojanovic et. al support that the higher level of abstraction that can be achieved through MDSD can significantly support agile principles \cite{stojanovic2003component}. They discuss the separation of concerns that can be achieved with the use of components \cite{stojanovic2003component}, a point that is also raised by Corcoran, who talks about MDSD experiences at Ericsson \cite{selic2010modelling}. He states that two major complexities exist within the large-scale organization that is Ericsson; firstly, the complexity with regard to the number of people involved in a project, and secondly the domain complexity \cite{selic2010modelling}. Corcoran states that by incorporating a modeling based development paradigm, the problems arising from these complexities are solved \cite{selic2010modelling}. 

Furthermore, MDSD and AD both focus on the development of software that can be understood and validated by the involved stakeholders and end users \cite{stahl2006model}. Stahl and Völter identify that MDSD does not contradict the agile principle of individuals and interactions over processes and tools \cite{stahl2006model}. An MDSD team is encouraged to establish their development process without the constraint of a document-centric approach \cite{stahl2006model}. Additionally, diagrams are considered the central artifact in MDSD, functioning both as working software and comprehensive documentation \cite{stahl2006model}. This inherent property of MDSD satisfies the agile principles of working software over comprehensive documentation as well as customer collaboration over contract negotiation, since the model diagrams function as the means for communication with the customer \cite{stahl2006model}. To support the argument that MDSD can help to scale agile practices, Stahl and Völter suggest that the agile principle of response to change over following a plan is facilitated with MDSD by the automatic implementation of a change in multiple parts of a system \cite{stahl2006model}.



\end{document}
