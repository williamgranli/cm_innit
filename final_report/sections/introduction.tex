\documentclass[fina_report_innit.tex]{subfiles}
\begin{document}

\section{Introduction}

The purpose of this paper is to determine whether an organisation's development paradigm should be the basis for deciding which process and design changes to adopt. To explore this area, we focus on process and design practices that have been introduced at Ericsson EPG.
\\

The process practices we researched were One Track, cross-functional teams and continuous integration. One Track (OT) is a development strategy for use with a version control system, the goal of which is to have only one system version, as opposed to having numerous development branches \cite{hribar2008first}. The use of Cross-functional teams (CFT) is a process, whereby project groups are assembled with members from different parts of an organisation with different skill-sets \cite{henke1993perspective}\cite{ghobadi2011challenges}. Usually consisting of eight to ten members, CF teams are thought to be innovative, creative \cite{ghobadi2011challenges} and by some, to be the single most important factor for success of software development \cite{marchwinski2000technical}. Continuous integration (CI) is a method of rapid and continuous integration commonly involving the use of automated builds and testing \cite{sommerville10software}.  
\\

The design practices we focused on were unit testing and component testing. Unit testing (UT) is the process of testing individual program units with various input parameters, where units this case are program methods or object classes. Similarly, component testing (CT) involves the testing of whole components and can often be a collection of smaller unit tests. The aforementioned practices are applied in both Model Driven Software Development (MDSD) and manual coding paradigms at Ericsson EPG.
\\

MDSD aims to discover abstractions at the domain level and makes it possible to implement these with formal modeling \cite{stahl2006model}. The developed models function abstractly and formally at the same time, serving not only documentation purposes, but also automatically generating code \cite{stahl2006model}. In juxtaposition to this, we use the term manual coding to describe software development practices which do not use MDSD. Both of these paradigms are employed by Ericsson, and more specifically by the Evolved Packet Gateway (EPG) department at Ericsson. This leads us to our research question: how have the individual changes in process and design practices at EPG affected MDSD compared to manual coding.
\\

The paper is divided into seven sections. Section II will provide a background into what and how both Ericsson and EPG work. Section III is the related work section, where we will present previous research on core concepts, theories and techniques which are used in both MDSD and manual coding. Section IV highlights the way change is conducted at Ericsson, while at the same time relating it to common change management theories. Section V describes the research approach, detailing the methodology used and what types of approach has been undertaken. Section VI is the results analysis where we provide initial findings along with the actual survey data. In our discussion, in section VII, we will deliberate upon the results and data collected, draw conclusions and argue for the reasons given. Finally, we finish with a concise description of the purpose and results of our research.

\end{document}
