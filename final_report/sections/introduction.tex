\documentclass[fina_report_innit.tex]{subfiles}
\begin{document}

\section{Introduction}
objective: to find which of the process and design activities are better suited to MDSD/MC
research question: have there been differences in how changes in process and design activities at EPG have affected MDSD groups compared to groups MC groups?
\subsection{FDD and OT intro}
One Track [ftriauotasld] is a development strategy which is closely related to feature-driven development [adoswa]. Feature-driven development's aim is to deliver software in small tangible units, where the units are features of the system [adoswa]. One Track mainly concerns the development strategy in terms of the version control system, and that the goal is to have only one system version, as opposed to having numerous development branches. 

\subsection{MDSD}
Model Driven Software Development (MDSD) is a programming paradigm which aims to discover abstractions at the domain level and make it possible to implement them with formal modeling [mdsden]. Effectively, the developed models function as abstract and formal simultaneously, serving not only documentation purposes, but are also considered equal to code which can be automatically implemented [mdsden].

\subsection{Unit Testing}
Unit testing is the process of testing individual program components with various input parameters, where units are in this case program methods or object classes.
For effective unit testing, tests should be automated when possible.  Frameworks such as jUnit makes automated testing approachable through graphical interfaces and generic test classes. An automated unit test consists of a setup part, where you provide the input variables and expected output, a call part where you call the relevant method or object, and lastly an assertion part where you compare the result to the expected outcome [SE]. 

\subsection{Continuous Integration}
With Agile methods ambition to do frequent releases it becomes necessary to have a way of doing integration without it being a long and time consuming process, which it often is in more traditional processes. Continuous Integration(CI) is a method of integrating all the components of a piece of software into one whole much more often than traditionally[se]. This method includes automated builds and testing. Since the tseting is automated and carried out daily or more often problems will be discovered early, that is the main benefit of CI. 

\subsection{Cross-Functional Teams}
The idea of XFT is to assemble project groups with a specific skill set from multiple departments within an organisation. Members with different skill sets, and with different identities, will, when assembled and managed correctly, be beneficial for all participants. Teams usually span from eight to ten people [XFT:CFTGCPI pp.219, XFT:CCFSDTCS pp.27] with the intention of promoting creativity and innovation [XFT:CCFSDTCS pp.27]. XFT is by some considered the single most important factor for software development success [XFT:TTCRICFT pp.69].

\end{document}
