\documentclass[fina_report_innit.tex]{subfiles}
\begin{document}

\section{Introduction}

Model Driven Software Development (MDSD) is a programming paradigm which aims to discover abstractions at the domain level and make it possible to implement them with formal modeling \cite{stahl2006model}. Effectively, the developed models function as abstract and formal simultaneously, serving not only documentation purposes, but are also considered equal to code which can be automatically implemented \cite{stahl2006model}. In juxtaposition to this, we use the term Manual Coding to describe software development practices which do not use MDSD. Both of these techniques are employed by Ericsson and more specifically the Evolved Packet Gateway (EPG) department at Ericsson.
\\

A brief literature review was also conducted and the results of which are described in the related work section. The intention of the review was to introduce many core concepts, theories and techniques which are used in both MDSD and Manual Coding split into two groups; design and process activities. The review was performed with a goal in mind; are any of these design and process activities better suited for MDSD or Manual Coding?
\\

The process activities we researched were One Track, Cross Functional Teams and Continuous Integration. Continuous Integration (CI) is a method of rapid and continuous integration commonly involving the use of automated builds and testing[softwareEngineering]. One Track (OT) is a development strategy for use with a version control system the goal of which is to have only one system version, as opposed to having numerous development branches \cite{hribar2008first}. The use of Cross Functional Teams (CF) is a process, whereby project groups are assembled with members from different parts of an organisation with different skill-sets \cite{henke1993perspective} pp.219, \cite{ghobadi2011challenges} pp.27. Usually consisting of eight to ten members, CF teams are thought to be innovative, creative \cite{ghobadi2011challenges} pp.27 and by some, to be the single most important factor for success of software development \cite{marchwinski2000technical} pp.69.
\\

Lastly, for design activities we have researched the use of Unit Testing and Component Testing. Unit Testing (UT) is the process of testing individual program components with various input parameters, where units are in this case program methods or object classes, similarly, Component Testing (CT) involves the testing of whole components and can often be a collection of smaller unit tests.
\\

This leads us to our main research question and the purpose of our paper; how have individual the changes in process and design activities at EPG affected MDSD groups compared to Manual Coding Groups.
\\

The paper is divided into 7 sections, starting with an introduction to Ericsson which will provide a background into what and how both Ericsson and EPG work. Following this is the related work section where we will introduce previous research which attempts to find supporting evidence for our research questions. Change at Ericsson comes afterwards, its main goal is to highlight the way change is conducted at Ericsson, while at the same time relating it to common change management theories[REFS HERE? - CM PAPERS USED]. The next section describes the research approach, detailing the methodologies used and what qualitative and quantitative approaches have been undertaken. Logically, the next section is the results analysis where we provide initial findings along with the actual data. The following discussion section is where we will deliberate upon the results and data collected, draw conclusions and argue for the reasons given. Finally, we finish with a concise description of the purpose and results of our research.

\end{document}
