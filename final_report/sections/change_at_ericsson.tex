\documentclass[fina_report_innit.tex]{subfiles}
\begin{document}

\section{Change at Ericsson}



The changes in Ericsson have been implemented in different ways and have emerged for different reasons. etc etc [[]]
It should be said that all changes share attributes of both top/bottom and epi/cont


\subsection*{Top-down Change at Ericsson}
Ericsson’s change strategy for top-down changes follows a four-stage process which can be described as follows: 1. Managers or architects realize that the opportunity and need for change. 2. A team is set up to plan the change and to tailor it to the EPG organisation
3. The change is evaluated and tested on pilot teams 4. Change is deployed EPG-wide [4]. Ericsson’s change process can be likened with Lewin’s 3-step model [6] which consists of the stages Unfreezing, Moving and Refreezing [6]. The initial stage of Ericsson’s process would correspond to the Unfreezing stage in Lewin’s model, which is described as a state where individuals’ equilibrium are shifted and opportunity for change in behaviour is presented [6]. The Moving stage of Lewin’s model correlates with [[to?]] with the three last stages of the Ericsson process and is by Lewin described as a phase where motivation to learn and improve moves the organisation in a direction [6]. The last stage of Lewin’s model is the phase where the new behaviour is fortified and a new equilibrium is formed [6] also presents itself in the last stage of Ericsson’s process. 

Ericsson’s top-down changes share many attributes with what Weick and Quinn describe as episodic changes [7]. Episodic changes are described as “infrequent, discontinuous and intentional” [7]. This can be seen for Ericsson’s change to increase UT and CT, where the reason for change were the increased costs of black box hardware testing [4]. The plan of implementing increased UT and CT was communicated from top management and it was mandatory for all developers [4]. The reason for changing to XFT was based on the notion of XFT being a better fit for the organization compared to functional teams [4]. 

The general idea within Ericsson seems to be that the changes have a goal, but that the goal is not strictly defined [4]. An example of this can be seen in Ericsson’s implementation of Scrum it was communicated that the goal was to change process, but after the change was implemented the exact usage of Scrum differed between the teams [4]. [[not sure if this is solid enough. also not sure if it can be connected to any CM theories]]

\subsection*{Bottom-up Change at Ericsson}
The change towards implementing OT and CI were initiated bottom-up, by the developers [4]. Many developers were dissatisfied with the current process for integration and the usage of development branches [4]. The largest issue was the long round-trip time (from the start of the development of a feature to the commit of a feature) [4]. The main reason for this was the time-consumingness [[ok word?]] of integration and testing development branches when they were not in sync with the master branch [4]. The characteristic of these changes share many features with what Van De Ven and Poole describe as the teleological motor [8]. Changes that are driven by the teleological motor are mainly driven by the displeasure [[synonym for discontent, ok?]] of the current status quo in an organization [8]. Teleological changes also don’t follow a set of predefined steps that are immutable which corresponds to the early phases of Ericsson’s implementation of OT and CI [4] [8]. One can, however, assume that the organization-wide implementation of OT and CI followed EPG’s standard top-down strategy, but the study of this has not been included in our research [[is this ok to say?]].



\end{document}
