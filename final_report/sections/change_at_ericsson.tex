\documentclass[final_report_innit.tex]{subfiles}
\begin{document}

\section{Change at Ericsson}

The changes in Ericsson have been implemented in different ways and have emerged for different reasons. etc etc [[]]
It should be said that all changes share attributes of both top/bottom and epi/cont

\subsection*{Top-down Change at Ericsson}
Ericsson’s change strategy for top-down changes follows a four-stage process which can be described as follows: 1. Managers or architects realise that the opportunity and need for change. 2. A team is set up to plan the change and to tailor it to the EPG organisation
3. The change is evaluated and tested on pilot teams 4. Change is deployed EPG-wide [4]. Ericsson’s change process can be likened with Lewin’s 3-step model [6] which consists of the stages Unfreezing, Moving and Refreezing [6]. The initial stage of Ericsson’s process would correspond to the Unfreezing stage in Lewin’s model, which is described as a state where individuals’ equilibrium are shifted and opportunity for change in behaviour is presented [6]. The Moving stage of Lewin’s model correlates with [[to?]] with the three last stages of the Ericsson process and is by Lewin described as a phase where motivation to learn and improve moves the organisation in a direction [6]. The last stage of Lewin’s model is the phase where the new behaviour is fortified and a new equilibrium is formed [6] also presents itself in the last stage of Ericsson’s process. 

Ericsson’s top-down changes share many attributes with what Weick and Quinn describe as episodic changes [7]. Episodic changes are described as “infrequent, discontinuous and intentional” [7]. This can be seen for Ericsson’s change to increase UT and CT, where the reason for change were the increased costs of black box hardware testing [4]. The plan of implementing increased UT and CT was communicated from top management and it was mandatory for all developers [4]. The reason for changing to XFT was based on the notion of XFT being a better fit for the organisation compared to functional teams [4]. 

The general idea within Ericsson seems to be that the changes have a goal, but that the goal is not strictly defined [4]. An example of this can be seen in Ericsson’s implementation of Scrum it was communicated that the goal was to change process, but after the change was implemented the exact usage of Scrum differed between the teams [4]. [[not sure if this is solid enough. also not sure if it can be connected to any CM theories]]

\subsection*{Bottom-up Change at Ericsson}
The change towards implementing OT and CI were initiated bottom-up, by the developers [4]. Many developers were dissatisfied with the current process for integration and the usage of development branches [4]. The largest issue was the long round-trip time (from the start of the development of a feature to the commit of a feature) [4]. The main reason for this was the time-consumingness [[ok word?]] of integration and testing development branches when they were not in sync with the master branch [4]. The characteristic of these changes share many features with what Van De Ven and Poole describe as the teleological motor [8]. Changes that are driven by the teleological motor are mainly driven by the displeasure [[synonym for discontent, ok?]] of the current status quo in an organisation [8]. Teleological changes also don’t follow a set of predefined steps that are immutable which corresponds to the early phases of Ericsson’s implementation of OT and CI [4] [8]. One can, however, assume that the organisation-wide implementation of OT and CI followed EPG’s standard top-down strategy, but the study of this has not been included in our research [[is this ok to say?]].

\subsection*{Measuring/Driving/Restraining Change at Ericsson}
Lewin has introduced the concepts of driving and restraining forces, which are opposing forces that together uphold a stable state in an organisation [6]. According to our interview, the main force which restrains change at Ericsson is the mindset of the employees [4]. The mindset often shows an inherent reluctancy to change [4]. A reason for this may be the hierarchical landscape of Ericsson [4]. To successfully gain enough mandate to push through a change, intimate knowledge of how the organisation works is needed [4]. Because of this, a lot of personal involvement and dedication is needed to convince managers of the change [4]. These factors can be likened with what Weick and Quinn refer to as an organisation's inertia [7]. The organisation's reluctancy to change has also been described in a case study conducted at Ericsson [9], where it was reported that management initially were opposing the change of moving to agile due to that agile development cycles are planned based on the working velocity of teams which meant that there was a restriction in how much can be prioritised for a given development cycle [9]. Another contributing factor that was made visible during the interview, was that previous attempts to change that had been based on scientific research had failed, due to the research being done in domains different to Ericsson's [4]. 

The main driver for [[of?]] change in EPG has been the perception that for example development processes other than Ericsson's is superior [4]. This conception has often been based on a external "hype", which was the case in the implementation of XFT [4]. Another stimulus for change in Ericsson emerges dialectically, as Van de Ven \& Poole refer to it [4] [8]. In the case of Ericsson, the enteties triggering each other are the different departments. Changes in EPG have been introduced based on the measured productivity the department, compared to other departments [4]. 

Kotter accentuates the importance of cementing the change after the initial implementation has been finished through proposing that the results of the change are communicated to the organisation [10]. In EPG, the success or failure of changes are measured through the change in feature flow, before and after the change has been implemented [4]. The feature flow is based on the requirements, and the requirements may vary in size which may cause the measurement to be unprecise [4]. The assumption is, however, that the results of the measurements will normalise over time [4].   


\subsection*{references william}
1. http://www.ericsson.com/thecompany 
2. http://www.ericsson.com/thecompany/company\_facts 
3. http://www.ericsson.com/ourportfolio/products/evolved-packet-gateway 
4. Jesper interview 
5. http://www-03.ibm.com/software/products/en/ratirhapfami 
6. Burnes, B. (2004). Kurt Lewin and the Planned Approach to Change: A Re-appraisal. Journal of Management Studies, 41(6), 977-1002.
7. Weick, K. E. \& Quinn, R. E. (1999). Organizational Change and Development. Annual review of psychology, 50, 361-386. 
8. Van de Ven, A. H. \& Poole, M. S. (1995). Explaining Development and Change in Organizations. Academy of Management Review, 20(3), 510-540. 
9. Karlstrom, D., \& Runeson, P. (2005). Combining agile methods with stage-gate project management. IEEE Software, 22(3), 43-49.
10. Kotter, P. J. (2007). Leading Change: Why Transformation Efforts Fail. Harvard Business Review, 85(1), 96-103. "
\end{document}
