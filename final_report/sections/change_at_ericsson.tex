\documentclass[final_report_innit.tex]{subfiles}
\begin{document}

\section{Investigating Change at Ericsson}
The recent changes that have been introduced at Ericsson have emerged due to different reasons. Changes have emerged from the top and bottom levels of the organisation. The changes included in this paper all share attributes of both top-down and bottom-up changes. 

\subsection{Top-down Change at Ericsson}
Ericsson’s change strategy for top-down changes follows a four-stage process which can be described as follows: (1) Managers or architects realise that there is is an opportunity and need a for change, (2) a team is set up to plan the change and to tailor it to the EPG organisation (3), the change is evaluated and tested on pilot teams (4), change is deployed EPG-wide (J. Derehag, personal communication, November 19, 2014). Ericsson’s change process can be likened to Lewin’s 3-step model which consists of the stages (1) Unfreezing, (2) Moving and (3) Refreezing \cite{burnes2004kurt}. The initial stage of Ericsson’s process corresponds to the Unfreezing stage in Lewin’s model, which is described as a state where individuals’ equilibrium is shifted and opportunity for change in behaviour is presented \cite{burnes2004kurt}. The Moving stage of Lewin’s model correlates to the last three stages of Ericsson's process, and is by Lewin described as a phase where motivation to learn and improve moves the organisation toward a certain direction \cite{burnes2004kurt}. The last stage of Lewin’s model is the phase where the new behaviour is fortified and a new equilibrium is formed \cite{burnes2004kurt}, which also presents itself in the last stage of Ericsson’s process. 
\\

Ericsson’s top-down changes share many attributes with what Weick and Quinn describe as episodic changes \cite{weick1999organizational}. Episodic changes are described as “infrequent, discontinuous and intentional” \cite{weick1999organizational}. This can be seen in Ericsson’s change to increase UT and CT, where the reason for change was the increased costs of black box hardware testing (J. Derehag, personal communication, November 19, 2014). The plan to increase UT and CT was communicated from top management and was mandatory for all developers (J. Derehag, personal communication, November 19, 2014). The reason for changing to CFT was based on the notion that CFT would be a better fit for the organisation compared to functional teams (J. Derehag, personal communication, November 19, 2014). 
\\

The general idea within Ericsson seems to be that the changes have a goal, but that the goal is not strictly defined (J. Derehag, personal communication, November 19, 2014). An example of this can be seen in Ericsson’s implementation of Scrum. It was communicated that the goal was to change the way of working, but after the change was implemented, the exact usage of Scrum differed between the teams (J. Derehag, personal communication, November 19, 2014). 

\subsection{Bottom-up Change at Ericsson}
The changes toward implementing OT and CI were initiated bottom-up, by the developers (J. Derehag, personal communication, November 19, 2014). Many developers were dissatisfied with the current process of integration and the usage of development branches (J. Derehag, personal communication, November 19, 2014). The largest issue was the long round-trip time (from the start of the development of a feature to the commit of a feature) (J. Derehag, personal communication, November 19, 2014). The main reason for this was the time-consumption of integration and testing development branches when they were not synchronised with the master branch (J. Derehag, personal communication, November 19, 2014). The characteristics of these changes share many features with what Van De Ven and Poole describe as the teleological motor \cite{van1995explaining}. Changes that are driven by the teleological motor are mainly driven by the dissatisfaction with the current status quo in an organisation \cite{van1995explaining}. Teleological changes also do not follow a set of predefined steps that are immutable, which corresponds to the early phases of Ericsson’s implementation of OT and CI (J. Derehag, personal communication, November 19, 2014) \cite{van1995explaining}. One can, however, assume that the organisation-wide implementation of OT and CI followed EPG’s standard top-down strategy, but this is not within the scope of our research.

\subsection{Accomplishment of Change at Ericsson}
Lewin introduces the concepts of driving and restraining forces, which are opposing forces that together uphold a stable state in an organisation \cite{burnes2004kurt}. According to our interview, the main force which restrains change at Ericsson is the mindset of the employees (J. Derehag, personal communication, November 19, 2014). The mindset often shows an inherent reluctancy to change (J. Derehag, personal communication, November 19, 2014). A reason for this may be the hierarchical landscape of Ericsson (J. Derehag, personal communication, November 19, 2014). To successfully gain enough mandate to push through a change, intimate knowledge of how the organisation works is needed (J. Derehag, personal communication, November 19, 2014). Due to this, a lot of personal involvement and dedication is needed to convince managers of the change (J. Derehag, personal communication, November 19, 2014). These factors can be likened to what Weick and Quinn refer to as an organisation's inertia \cite{weick1999organizational}. The organisation's reluctancy to change has also been described in a case study conducted at Ericsson \cite{karlstrom2005combining}, where it was reported that management were initially opposing the change of moving to agile, due to the fact that agile development cycles are planned based on the working velocity of teams. This means that there is a restriction in how much can be prioritised for a given development cycle \cite{karlstrom2005combining}. Another contributing factor that was indicated during the interview, was that previous attempts to change that had been based on scientific research had failed, due to the research being conducted in domains different to Ericsson's (J. Derehag, personal communication, November 19, 2014). 
\\

The main driver of change in EPG has been the perception that development processes or tools other than the ones currently in use at Ericsson are superior (J. Derehag, personal communication, November 19, 2014). This conception has often been based on an external ``hype'', which was the case in the implementation of CFT (J. Derehag, personal communication, November 19, 2014). Another stimulus for change at Ericsson emerges dialectically \cite{van1995explaining}. In the case of Ericsson, the entities triggering each other are the different departments. Changes in EPG have been introduced based on the measured productivity between the organisation's departments. (J. Derehag, personal communication, November 19, 2014)
\\

Kotter accentuates the importance of cementing the change after the initial implementation has been finished, by proposing that the results of the change are communicated to the organisation \cite{kotter1995leading}. In EPG, the success or failure of changes are measured through the change in feature flow, before and after the change has been implemented (J. Derehag, personal communication, November 19, 2014). The feature flow is based on the requirements, which vary in size, potentially causing the measurement to be imprecise (J. Derehag, personal communication, November 19, 2014). The assumption is, however, that the results of the measurements will normalise over time (J. Derehag, personal communication, November 19, 2014).   


\end{document}
