\documentclass[ProjectPlan_innit.tex]{subfiles}

\begin{document}
Ericsson have made many changes the last five years which has affected groups in different ways. However, how the groups are affected is not known since no metrics are kept. Therefore, our research shall be inductive by means of for future changes pointing out how easy it has been to adapt and implement previous changes.

\smallskip
We are interested in knowing both how changes has affected group contributions and achievements but also how each individual has reacted upon it.

\smallskip
Our key metrics are;
\begin{enumerate}
	\item productivity (measured by the potential of meeting deadlines and deliver on time),
	\item product quality,
	\item personal fulfilment, and
	\item ease of implementation.
\end{enumerate}

\smallskip
A brief research has already been carried out with our contact person at Ericsson. Our initial area of intereset was broken down due to a considerable amount of work load, thus, narrowing down our subject to comparing five changes within two focus groups (as mentioned, MDD and manual coding teams).

\smallskip
These five changes are;
\begin{enumerate}
	\item Continuous Integration,
	\item Cross-Functional Teams,
	\item Feature Toggles and One Track,
	\item Unit Testing, and
	\item Component Testing.
\end{enumerate}

\subsection{Tools}
Data shall be collected through a questionnaire and analysed with a spreadsheet compatible software, allowing grouping and filtering of raw results. We have selected Google Sheets to simplify collaboration withing our team. The questionnaire is composed through Google Form which also support graphical statistics of the result.

\subsection{Quantitative data collection}
We are not interested in how for example an MDD team uses Unit Testing in their daily work. We are just interested in the change itself, which implies that the questionnaire is generic enough to verify if groups and individuals have reached satisfactory levels for the metrics or not (e.g. the metric indicating personal fulfilment reaching any positive level). 

\subsection{Qualitative data collection}
Whereas quantitative data collection will give us a broad spectrum of conceptions, interviews shall enable us to strengthen theories and fill in the pot holes of our research. As of today, the final interview material is not produced, but shall be based on uncertainties from the questionnaire.

\smallskip
Future documentation will explain how verification would be carried out, and also elaborating how to analyse and compile raw data into result.

\end{document}