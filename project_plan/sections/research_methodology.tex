\documentclass[ProjectPlan_innit.tex]{subfiles}

\begin{document}
Ericsson have made many changes the last five years which has affected groups in different ways. However, how the groups are affected is not known since no metrics are kept. Therefore, our research shall be inductive by means of for future changes pointing out how easy it has been to adapt and implement previous changes.

\hspace{0pt}\\
We are interested in knowing both how changes has affected group contributions and achievements but also how each individual has reacted upon it. Therefore, our key metrics consist of;
\list
	\item productivity (measured by the potential of meeting deadlines and deliver on time),
	\item product quality,
	\item personal fulfilment, and
	\item ease of implementation.
\endlist

\hspace{0pt}\\ 
A brief research has already been carried out with our contact person at Ericsson. Our initial area of intereset was broken down due to a considerable amount of work load, thusly, narrowing down our subject to comparing five changes within two focus groups (as mentioned, MDD and manual coding teams).

\hspace{0pt}\\ 
These five changes are;
\list{enumerate}
	\item Continuous Integration,
	\item Cross-Functional Teams,
	\item Feature Toggles & One Track,
	\item Unit Testing, and
	\item Component Testing.
\end{enumerate}

\hspace{0pt}\\ 

\subsection{Tools}
Data shall be collected through a questionnaire and analysed with a spreadsheet compatible software, allowing grouping and filtering of raw results. The questionnaire is composed through Google Form which also support graphical statistics of the result. Preferably Google Sheets shall be used for easier collaboration within our group.

\subsection{Quantitative data collection}
We are not interested in how i.g. an MDD team uses Unit Testing in their daily work. We are just interested in the change itself, which implies that the questionnaire is generic enough 
generic because not interested in knowing details of mdd or manual. interested in the change itself

--------------------------------------------------------------------------------------------
\subsection{Initial quantitative data collection}
The quantitative data collection phase shall present responses and perceptions of previously mentioned change areas and on a collective level show if the key metrics reaches satisfactory levels.

\subsection{Complementary qualitative data collection}
Since quantitative data gives us a broad indication of differences between the focus groups, interviews shall be created to narrow our focus toward some areas of interest. 

\hspace{0pt}\\ Future documentation will explain how verification would be carried out, and also elaborating how to analyse and compile raw data into result.

\end{document}