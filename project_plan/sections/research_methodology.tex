\documentclass[ProjectPlan_innit.tex]{subfiles}

\begin{document}
As explained in the previous chapter, collecting data will be carried out mainly through a survey. To get as precise data as possible we won't differ the survey dependant on if groups are working with manual coding or MDD. Therefore, questionnaires shall be generic enough to find opinions regarding productivity, product quality and personal fulfillment for both focus groups. Ericsson lacks proper metrics for mentioned criterias. Therefore, this research can be used inductively when future changes are taking place. Additionally, data will be collected through interviews.

\hspace{0pt}\\ Note that at this early stage during research, surveys has not yet been formed. Therefore, extensive ellaborations on how the data shall be used is not possible to supply. 

\subsection{Initial quantitative data collection}
At Ericsson, employees working in an MDD environment might respond different to changes compared to employees developing through manual coding. The quantitative data collection phase shall present responses and perceptions of previously mentioned change areas and on a collective level show if the key metrics reaches satisfactory levels. Additionally, this survey will contain control questions to verify the result.

\subsection{Complementary qualitative data collection}


Whereas quantitative data will give us a broad indication of differences between the focus groups, interviews shall be created to narrow our focus toward areas of interest. 

\hspace{0pt}\\ Future documentation will explain how verification would be carried out, and also ellaborating how to analyse and compile raw data into result.

\end{document}